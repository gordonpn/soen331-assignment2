\documentclass[12pt]{article}
\usepackage[top=1in,bottom=1in,left=1in,right=1in]{geometry}
\usepackage{alltt}
\usepackage{array}	
\usepackage{graphicx}
\usepackage{tabularx}
\usepackage{verbatim}
\usepackage{setspace}
\usepackage{listings}
\usepackage{amssymb,amsmath, amsthm}

\title{SOEN331: Introduction to Formal Methods\\for Software Engineering\\
Assignment 2 on Extended Finite State Machines}
\author{Gordon Pham-Nguyen, Ian Philips, Tabesh Haidary, Jeffrey Li}
\date{\today}

\begin{document}
\begin{spacing}{1.5}

\maketitle

\newpage

\section{Driver-less car system formal specification}

\noindent The EFSM of the driver-less car system is the tuple $S = (Q, \Sigma_1, \Sigma_2, q_0, V, \Lambda)$, where\\

\noindent $Q = \{Idle~(Parked), Manual~Mode, Cruise~Mode, Panic~Mode\}$\\
\noindent $\Sigma_1 = \{State~engine, shut~off~engine, drive~signal, parked~signal, cruise~signal, manual~signal,\\ car~arrived~at~destination, panic~mode~off, unforeseen~event~or~panic~signal\}$\\
\noindent $\Sigma_2 = \{beep, stop~car, turn~on, hazard~lights, turn~off~hazard~lights\}$\\
\noindent $q_0: Idle~(Parked)$\\
\noindent $V: \{\}$\\
\noindent $\Lambda$: Transition specifications\\
\indent 1. $\rightarrow Idle~(Parked)$\\
\indent 2. $Idle~(Parked) \xrightarrow {\text {cruise~signal [destination is set]}} Cruise~Mode$\\
\indent 3. $Idle~(Parked) \xrightarrow {\text {drive~signal}} Manual~Mode$\\
\indent 4. $Manual~Mode \xrightarrow {\text {cruise~signal [destination is set]}} Cruise~Mode$\\
\indent 5. $Cruise~Mode \xrightarrow {\text {car~arrived~at~destination}} Idle~(Parked)$\\
\indent 6. $Idle~(Parked) \xrightarrow {\text {cruise~signal [destination is not set] / beep}} Idle~(Parked)$\\
\indent 7. $Manual~Mode \xrightarrow {\text {parked~signal [car is stopped]}} Idle~(Parked)$\\
\indent 8. $Cruise~Mode \xrightarrow {\text {unforeseen~event~or~panic~signal / stop car ; turn on hazard lights}} Panic~Mode$\\
\indent 9. $Panic~Mode \xrightarrow {\text {panic~mode~off / turn off hazard lights}} Idle~(Parked)$\\
\indent 10. $Manual~Mode \xrightarrow {\text {cruise signal [destination is not set] / beep}} Manual~Mode$\\
\indent 11. $Cruise~Mode \xrightarrow {\text {manual signal}} Manual~Mode$\\
\indent 12. $Idle~(Parked) \xrightarrow {\text {shut off engine}} Exit$\\

\newpage

\noindent The EFSM of the car system in manual mode is the tuple $S = (Q, \Sigma_1, \Sigma_2, q_0, V, \Lambda)$, where\\

\noindent $Q = \{Manual~Mode, Break~Mode\}$\\
\noindent $\Sigma_1 = \{accelerate~engine, reduce~signal, break~signal, accelerate~signal\}$\\
\noindent $\Sigma_2 = \{faster~engine, slower~engine, 0-speed\}$\\
\noindent $q_0: Manual~Driving$\\
\noindent $V: \{\}$\\
\noindent $\Lambda$: Transition specifications\\
\indent 1. $\rightarrow Manual~Driving$\\
\indent 2. $Manual~Driving \xrightarrow {\text {accelerate~signal/faster~engine}} Manual~Driving$\\
\indent 3. $Manual~Driving \xrightarrow {\text {reduce~signal/slower~engine}} Manual~Driving$\\
\indent 4. $Manual~Driving \xrightarrow {\text {break~signal/0-speed}} Break~Mode$\\
\indent 5. $Break~Mode \xrightarrow {\text {accelerate~signal}} Manual~Mode$\\

\newpage

\noindent The EFSM of the car system in cruising mode is the tuple $S = (Q, \Sigma_1, \Sigma_2, q_0, V, \Lambda)$, where\\

\noindent $Q = \{Cruising, Tailing, Changing~Lane\}$\\
\noindent $\Sigma_1 = \{Car~detected~ahead, after~1~second, any~change~lane~signal, target~lane~signal, \\ panic~signal\}$\\
\noindent $\Sigma_2 = \{reduce~speed, maintain~current~speed, change~to~left~lane~signal, panic~signal\}$\\
\noindent $q_0: Cruising$\\
\noindent $V: \{\}$\\
\noindent $\Lambda$: Transition specifications\\
\indent 1. $\rightarrow Cruising$\\
\indent 2. $Cruising \xrightarrow {\text {Car~detected~ahead [distance under threshhold]/ reduce~speed}} Tailing$\\
\indent 3. $Tailing \xrightarrow {\text {after~1~second [distance under threshold]/ reduce~speed}} Tailing$\\
\indent 4. $Tailing \xrightarrow {\text {after~1~second [obstacle not moving or safe distance cannot be maintained]/ \\change~to~left~lane~signal}}\\ \indent     ~~~Changing~Lane$\\
\indent 5. $Tailing \xrightarrow {\text {after~1~second [distance is above or at threshold]/ maintain~current~speed}} Cruising$\\
\indent 6. $Cruising \xrightarrow {\text {any~lane~change~signal}} Changing~Lane$\\
\indent 7. $Changing~Lane \xrightarrow {\text {target~lane~signal}} Cruising$\\
\indent 8. $Changing~Lane \xrightarrow {\text {panic~signal/panic~signal}} Cruising$\\

\newpage

\noindent The EFSM of the car system in navigating mode is the tuple $S = (Q, \Sigma_1, \Sigma_2, q_0, V, \Lambda)$, where\\

\noindent $Q = \{Navigating\}$\\
\noindent $\Sigma_1 = \{After~1~second\}$\\
\noindent $\Sigma_2 = \{accelerate~signal, increase~speed, reduce~speed~signal, decrease~speed,
\\car~arrived~at~destination~signal, turn~right~at~next~intersection, turn~left~at~next~intersection, 
\\change~to~right-most~lane~signal, change~to~left-most~lane~signal\}$\\
\noindent $q_0: Navigating$\\
\noindent $V: \{\}$\\
\noindent $\Lambda$: Transition specifications\\
\indent 1. $\rightarrow Navigating$\\
\indent 2. $Navigating \xrightarrow {\text {after~1~second [left turn ahead]/ change~to~left-most~lane~signal}} Navigating$\\
\indent 3. $Navigating \xrightarrow {\text {after~1~second [right turn ahead]/ change~to~right-most~lane~signal}} Navigating$\\
\indent 4. $Navigating \xrightarrow {\text {after~1~second [Destination ahead]/ change~to~right-most~lane~signal}} Navigating$\\
\indent 5. $Navigating \xrightarrow {\text {after~1~second [turn left]/ turn~left~at~next~intersection}} Navigating$\\
\indent 6. $Navigating \xrightarrow {\text {after~1~second [turn right]/ turn~right~at~next~intersection}} Navigating$\\
\indent 7. $Navigating \xrightarrow {\text {after~1~second [arrived at destination]/ car~arrived~at~destination~signal}} Navigating$\\
\indent 8. $Navigating \xrightarrow {\text {after~1~second [current speed more than road speed + 5\%]/ reduce~speed~signal; decrease~speed}} 
\\\indent~~~Navigating$\\
\indent 9. $Navigating \xrightarrow {\text {after~1~second [current speed more than road speed - 5\%]/ accelerate signal; increase speed}} 
\\\indent~~~Navigating$\\

\newpage

\noindent The EFSM of the car system in lane changing mode is the tuple $S = (Q, \Sigma_1, \Sigma_2, q_0, V, \Lambda)$, where\\

\noindent $Q = \{Merging Lane\}$\\
\noindent $\Sigma_1 = \{After~1~second\}$\\
\noindent $\Sigma_2 = \{target~lane~signal, obstacle~ahead~or~cannot~change~lane, panic~signal, change~to~right~lane,
\\\noindent change~to~left~lane, change~right~lane~signal, change~left~lane~signal\}$\\
\noindent $q_0: Merging Lane$\\
\noindent $V: \{Lanes\}$\\
\noindent $\Lambda$: Transition specifications\\
\indent 1. $\rightarrow Merging Lane$\\
\indent 2. $Merging Lane \xrightarrow {\text {after~1~second [Lanes == 0]/ target~lane~signal}} Merging Lane$\\
\indent 3. $Merging Lane \xrightarrow {\text {after~1~second [obstacle~ahead~or~cannot~change~lane]/ panic~signal}} Merging Lane$\\
\indent 4. $Merging Lane \xrightarrow {\text {after~1~second [Lanes $\textgreater$ 0 $\wedge$ right lane is open]/ (change~to~right~lane ; Lanes$\textendash~\textendash$)}} Merging Lane$\\
\indent 5. $Merging Lane \xrightarrow {\text {after~1~second [Lanes $\textgreater$ 0 $\wedge$ right lane is not open]/ change~right~lane~signal}} Merging Lane$\\
\indent 6. $Merging Lane \xrightarrow {\text {after~1~second [Lanes $\textless$ 0 $\wedge$ left lane is open]/ (change~to~left~lane ; Lanes++)}} Merging Lane$\\
\indent 7. $Merging Lane \xrightarrow {\text {after~1~second [Lanes $\textless$ 0 $\wedge$ left lane is not open]/ change~left~lane~signal}} Merging Lane$\\

\newpage

\section{UML state diagrams}

% \noindent The UML state diagram is shown in Figure~\ref{fig:metro-fig}.

% \begin{figure}[h!]
% 	\centering
% 		\includegraphics[width=0.8\textwidth]{./figures/eps/metro.eps}
% 		  \caption{Metro.}
%   \label{fig:metro-fig}
% \end{figure}

\end{spacing}
\end{document}
